
%->8-------------------------------------------------------------------------------------------8<-%
\usepackage[brazil]{babel}
\usepackage[utf8]{inputenc}
\usepackage[T1]{fontenc}
%->8-------------------------------------------------------------------------------------------8<-%

%->8-------------------------------------------------------------------------------------------8<-%
\usetheme{Darmstadt}
%->8-------------------------------------------------------------------------------------------8<-%

%->8-------------------------------------------------------------------------------------------8<-%
\usepackage{graphicx}
\usepackage{float}
%->8-------------------------------------------------------------------------------------------8<-%

%->8-------------------------------------------------------------------------------------------8<-%
\usepackage{geometry}
\geometry{left=0.4cm, right=0.4cm}
\usepackage{csquotes}
\usepackage{amsmath, amssymb}
\usepackage{upquote}
\usepackage{listings}
\usepackage{hyperref}
\setbeamertemplate{navigation symbols}{}
%->8-------------------------------------------------------------------------------------------8<-%

%->8-------------------------------------------------------------------------------------------8<-%
\usepackage[style=numeric, backend=biber]{biblatex}
\addbibresource{repertoire.bib}
\graphicspath{{figures/}}
%->8-------------------------------------------------------------------------------------------8<-%

%->8-------------------------------------------------------------------------------------------8<-%
\lstnewenvironment{code}[1][]{
  \lstset{
    basicstyle=\ttfamily,
    columns=flexible,
    breaklines=true,
    breakatwhitespace=true,
    frame=none,
    basewidth=0.5em,
    aboveskip=13pt,
    belowskip=0pt,
    #1
  }
}{}

\lstnewenvironment{styledoutput}[1][]{
  \lstset{
    basicstyle=\ttfamily\itshape,
    breaklines=true,
    breakatwhitespace=true,
    frame=none,
    aboveskip=0pt,
    #1
  }
}{}

\newcommand{\userinput}[1]{
  \noindent
  \texttt{\textit{>}}
  \texttt{#1}
}

\newcommand{\useroutput}[1]{\textit{\texttt{#1}}}
%->8-------------------------------------------------------------------------------------------8<-%

%->8-------------------------------------------------------------------------------------------8<-%
\title{Implementando Operadores Genéricos}
\subtitle{Uma Comparação entre Scheme e Lua}
\author{João Victor Lopez Pereira}
\institute{Instituto de Computação -- UFRJ}
\date{\today}
%->8-------------------------------------------------------------------------------------------8<-%
